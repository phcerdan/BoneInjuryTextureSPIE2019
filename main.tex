\documentclass[10pt,aspectratio=169]{beamer}
\usepackage{pgfpages}

% ---------------- Support for notes ------------------------------
% Add an empty note to every slide
% https://tex.stackexchange.com/questions/11708/run-macro-on-each-frame-in-beamer
% \setbeameroption{hide notes} % Only slides (default)
% \setbeameroption{show only notes} % Only notes
%\setbeameroption{show notes on second screen=right} % Both
\makeatletter
\def\beamer@framenotesbegin{% at beginning of slide
  \gdef\beamer@noteitems{}%
  \gdef\beamer@notes{{}}% used to be totally empty.
}
\makeatother
% Now, use a pdf reader able to read this:
% -----------------------------------------------------------------
% iOS: Skim: https://gist.github.com/andrejbauer/ac361546ac2186be0cdb
% 1. Generate just the presentation (hide notes) and save to slides.pdf
% 2. Generate onlt the notes (show only nodes) and save to notes.pdf
% 3. With Skim open both slides.pdf and notes.pdf
% 4. Click on slides.pdf to bring it to front.
% 5. In Skim, under "View -> Presentation Option -> Synhcronized Noted Document"
%    select notes.pdf.
% 6. Now as you move around in slides.pdf the notes.pdf file will follow you.
% 7. Arrange windows so that notes.pdf is in full screen mode on your laptop
%    and slides.pdf is in presentation mode on the projector.
% -----------------------------------------------------------------
% Linux: pdfpc: https://tex.stackexchange.com/questions/84622/is-there-a-specialized-pdf-viewer-for-latex-beamer-presentations-on-linux
% example: pdfpc main.pdf --notes=right
% ---------------- End support for notes --------------------------

\graphicspath{{./figures/}{./logos/}}
\usetheme[progressbar=frametitle, sectionpage=none]{metropolis}
\usepackage{appendixnumberbeamer}
\usepackage{siunitx}
\usepackage{framed}
\usepackage{color}
\usepackage{hyperref}
\input{xincludegraphics.tex}
\hypersetup{
colorlinks, urlcolor=blue, linkcolor={blue!10!black},
citecolor={blue!50!black}
}
%\usepackage[style=authortitle,backend=bibtex]{biblatex}
\usepackage[backend=biber,bibencoding=utf8,style=verbose-inote,citestyle=authortitle]{biblatex}
\addbibresource{references.bib}
%
% metropolis theme: https://github.com/matze/mtheme
%
\mode<presentation>
{
  \usetheme{metropolis}      % or try Darmstadt, Madrid, Warsaw, ...
  %\usecolortheme{default} % or try albatross, beaver, crane, ...
  %\usefonttheme{default}  % or try serif, structurebold, ...
  \setbeamertemplate{navigation symbols}{}
  \setbeamertemplate{caption}[numbered]
} 
\definecolor{kitwaregray}{HTML}{686868}
\definecolor{kitwarelightgray}{HTML}{EEEEEE}
\definecolor{kitwareblue}{HTML}{005F9E}
\definecolor{kitwaregreen}{HTML}{009D46}
\definecolor{kitwarewhite}{HTML}{FFFFFF}
\definecolor{kitwareblack}{HTML}{000000}
\definecolor{shadecolor}{HTML}{EEEEEE}

\setbeamercolor{progress bar}{fg=kitwaregreen}
% Theme colors are derived from these two elements
%\setbeamercolor{alerted text}{fg=}
\setbeamercolor{frametitle}{bg=kitwareblue}
\setbeamercolor{footline}{bg=kitwaregreen}
\setbeamerfont{frame numbering}{size=\small}
\setbeamertemplate{footline}{%
  \begin{beamercolorbox}[wd=\textwidth, sep=1ex]{footline}%
    %\usebeamerfont{page number in head/foot}%
    \usebeamertemplate*{frame footer}
    \hfill%
    \usebeamertemplate*{frame numbering}
  \end{beamercolorbox}%
}
% At the logo on top of the footline.
\addtobeamertemplate{footline}{\hfill\includegraphics[height=0.45cm]{klogo_new_crop}\hspace*{0.5em}\par}{}
%\setbeamertemplate{frame footer}{\includegraphics[width=.1\textwidth]{klogo_new_crop}}


\title{Methods for quantitative characterization of bone injury from computed-tomography images}
%\subtitle{Methods for quantitative characterization of bone injury from computed-tomography images}
\date{\today}
\author{Pablo Hernandez-Cerdan\textsuperscript{a}, Beatriz Paniagua\textsuperscript{a}, Jack Protero\textsuperscript{b}, J.S Marron\textsuperscript{b}, Eric Libingston\textsuperscript{c}, Ted Bateman\textsuperscript{c} and Matthew McCormick\textsuperscript{a}}
\institute{\textsuperscript{a} Kitware, Inc., NC, USA\newline\textsuperscript{b} Dept. of Statistics and Operations Research, University of North Carolina at Chapel Hill, NC, USA\newline\textsuperscript{c} Dept. of Biomedical Engineering, University of North Carolina at Chapel Hill, NC, USA}

\titlegraphic{
\begin{tikzpicture}[overlay, remember picture]
\node[at=(current page.south), anchor=south] {%
\includegraphics[width=.3\textwidth]{klogo}\hspace{2cm} \includegraphics[width=.3\textwidth]{unclogo}
};
\end{tikzpicture}
}
% \titlegraphic{\hfill\includegraphics[height=1.5cm]{logo.pdf}}

\begin{document}

\maketitle

% \begin{frame}{Table of contents}
%   \setbeamertemplate{section in toc}[sections numbered]
%   \tableofcontents[hideallsubsections]
% \end{frame}

\section{Introduction}

\begin{frame}
  \begin{columns}[onlytextwidth]
  \column{0.5\textwidth}
    \includegraphics[width=0.9\textwidth]{./logos/klogo.png}\\
    \vspace{0.1cm}
    % \metroset{block=fill}
    \begin{block}{\color{kitwareblue} Collaborative Software R\&D}
      \begin{itemize}%\itemsep0.2em
        \item Algorithms \& applications
        \item Software process \& infrastructure
        \item Support \& training
        \item Open source leadership
      \end{itemize}
    \end{block}
    \begin{block}{\color{kitwareblue} Supporting industry, government and academia}
    \end{block}
  \column{0.5\textwidth}
    \centering
    \includegraphics[width=0.9\textwidth]{./logos/kitware_pie.png}\\
    \includegraphics[width=0.9\textwidth]{./logos/kitware_itk_slicer_vtk_logos.png}
  \end{columns}

  \note[item]{Intro Kitware:}
\end{frame}

\begin{frame}[fragile]{Introduction}
\begin{itemize} \itemsep1em
\item Bone diseases are present in 50\% adults in the United States (CDC 2012)

\item Imaging provides a fast, scalable and non-invasive to examine bone structure. But the evaluation is often performed qualitatively or semi-quantitative.

\item Our goal is to present open source software tools to quantify musculoskeletal diseases through imaging.

\item These methods are tested on a pilot dataset of femurs scans obtained from a hemophilia-induced mouse models.
\end{itemize}

\note{
  \begin{itemize}
    \item CDC 2012: Centers for Disease control and Prevention: 50\% is 126 million (2012)
    \item Fractures and Osteoporosis are the biggest.
    \item Not many pre-clinical tools. Imaging is important for therapeutic approaches because easy translated into the clinical setting.
    \item However, the goal would be to offer quantitative analysis, instead of the current qualitatively or semi-quantitative.
  \end{itemize}
}
\end{frame}

\section{Materials}

\begin{frame}[fragile]{Materials}
\begin{columns}[onlytextwidth]
  \column{0.25\textwidth}
    \centering
    \includegraphics[width=0.99\textwidth]{figures/bone_materials_femur_highlight.png}\\
    \vspace{0.5cm}
    \includegraphics[width=0.99\textwidth]{figures/bone_materials_femur_3D.png}
  \column{0.75\textwidth}
    \centering
    \begin{itemize} \itemsep1em
        \item Three, 22 week old (skeletally mature), genetically modified, male mice.
        \item Right limb is healthy (control) and left limb has an induced wound.
        \item F8 gene knockout preventing the generation of protein coagulation factor VIII.
        \item Pathogenic pathway similar to hemophilia, presenting bleeding induced bone and joint damage.
    \end{itemize}
\end{columns}
\note{
  \begin{itemize}
    \item Right limb is control, left is wounded.
    \item Wound induced in the left limb. knee joint hemorraghe by puncturing the joint capsule with a needle
    \item Two weeks after, mice euthanized.
  \end{itemize}
}
\end{frame}

\begin{frame}[fragile]{Materials}
  \begin{itemize} \itemsep1em
    \item Images acquired using microCT, with resolution of $10$\si{\micro} (\si{\micro}CT80, Scanco)
    \item Trabecular bone at the proximal tibia, inferior to the growth plate
  \end{itemize}
  \vspace{0.5cm}
  \begin{columns}[onlytextwidth]
    \column{0.5\textwidth}
    \centering
    \includegraphics[width=4cm]{figures/bone_materials.png}
    \column{0.5\textwidth}
    \centering
    \includegraphics[width=3.7cm]{figures/bone_materials_3D.png}
  \end{columns}
  \note{
    acquistion parameters: (only if asked)
    \begin{itemize}
      \item FOV: 40 x 40 mm
      \item high resolution scanning mode
      \item 90 kV, 5 mA, 30.8 s, 125 $\mu$m voxel size, effective dose 114 $\mu$Sv.
    \end{itemize}
  }
\end{frame}

{
\setbeamercolor{background canvas}{bg=kitwarewhite}
\begin{frame}{Image Analysis Workflow}
\begin{columns}[onlytextwidth]
    \column{0.75\textwidth}
      \includegraphics[width=0.99\textwidth]{figures/Automated_Processing_Pipeline.png}
        %\includegraphics[width=0.99\textwidth]{figures/analysis_workflow.png}
    \column{0.25\textwidth}
        \centering
        \includegraphics[width=2.5cm]{logos/logo_slicer.png}\\
        \vspace{0.3cm}
        Data and all the steps for reproducibility at: \url{http://dx.doi.org/10.6084/m9.figshare.6947891}
\end{columns}
\end{frame}
}

\section{Methods}

\begin{frame}[fragile]{Quantitative Methods}
  \begin{columns}
    \column{0.55\textwidth}
    \centering
    \begin{itemize} \itemsep2em
      \item Texture Analysis: GLCM and GLRLM
      \item Bone Morphometry
      \item Shape Analysis: Iterative Closest Points (ICP)
    \end{itemize}
    \column{0.4\textwidth}
    \centering
    \includegraphics[width=0.8\textwidth]{./logos/logo_ITK.png}
  \end{columns}
\end{frame}

\begin{frame}[fragile]{Texture Analysis}
  \noindent We investigated two different types of textural features related to different aspects of the bone architecture:
  \vspace{0.1cm}
  % \metroset{block=fill}
  \begin{columns}
  \column{0.5\textwidth}
  \centering
  \begin{exampleblock}{Co-occurrence based-features:}
    \begin{itemize} \itemsep0.8em
      \item \cite{Haralick1973} [1973]
      \item Local neighborhoods and connectivities
      \item Grey Level Co-occurrence Matrix (GLCM)
    \end{itemize}
  \end{exampleblock}
  \column{0.5\textwidth}
  \centering
  \begin{exampleblock}{Run-length based-features:}
    \begin{itemize} \itemsep0.8em
      \item \cite{Galloway1975} [1975]
      \item Grey-level clusters
      \item Grey Level Run-Length Matrix (GLRLM)
    \end{itemize}
  \end{exampleblock}
  \end{columns}
\end{frame}

\begin{frame}{Texture Analysis: Co-occurrence Matrix (GLCM)}
  \begin{columns}[onlytextwidth]
    \column{0.6\textwidth}
    Eight co-occurrence based-features:\\
    \begin{itemize} \itemsep0.5em
      \item Energy
      \item Entropy \textbf{[a]} 
      \item Correlation \textbf{[b]}
      \item Inverse Difference Moment (IDM) \textbf{[c]} 
      \item Inertia \textbf{[d]}
      \item Cluster shade
      \item Cluster prominence
      \item Haralick’s correlation \textbf{[e]}
    \end{itemize}
    \column{0.4\textwidth}
    \begin{columns}
      \centering
      %\includegraphics[width=0.3\textwidth]{./Figures/hot_to_cold.png}\\
      \column{0.5\textwidth}
      \centering
      % \xincludegraphics[width=0.8\textwidth, label=a, labelbox]{./TextureMapsImages/GLCM_46_left_0_energy.png}\\
      \xincludegraphics[width=0.8\textwidth]{./TextureMapsImages/GLCM_46_left_grayscale.png}\\
      \vspace{0.3mm}
      \xincludegraphics[width=0.8\textwidth, label=b, labelbox]{./TextureMapsImages/GLCM_46_left_2_correlation.png}\\
      \vspace{0.3mm}
      \xincludegraphics[width=0.8\textwidth, label=d, labelbox]{./TextureMapsImages/GLCM_46_left_4_inertia.png}
      \column{0.5\textwidth}
      \centering
      \xincludegraphics[width=0.8\textwidth, label=a, labelbox]{./TextureMapsImages/GLCM_46_left_1_entropy.png}\\
      \vspace{0.3mm}
      \xincludegraphics[width=0.8\textwidth, label=c, labelbox]{./TextureMapsImages/GLCM_46_left_3_inverseDifference.png}\\
      \vspace{0.3mm}
      \xincludegraphics[width=0.8\textwidth, label=e, labelbox]{./TextureMapsImages/GLCM_46_left_7_haralicCorrelation.png}
    \end{columns}
  \end{columns}
\end{frame}

\begin{frame}{Texture Analysis: Run-Length Matrix (GLRLM)}
  \begin{columns}[onlytextwidth]
    \column{0.6\textwidth}
    Ten run-length based-features:
    \begin{itemize}
       \item Short run emphasis (SRE) \textbf{[a]}
       \item Long run emphasis (LRE) \textbf{[b]}
       \item Grey level non-uniformity (GLN) \textbf{[c]}
       \item Run length non-uniformity (RLN) \textbf{[d]}
       \item Low grey level run emphasis (LGRE) \textbf{[e]}
       \item High grey level run emphasis (HGRE)
       \item Short run low grey level emphasis (SRLGE)
       \item Short run high grey level emphasis (SRHGE)
       \item Long run low grey level emphasis (LRLGE)
       \item Long run high grey level emphasis (LRHGE)
    \end{itemize}
    \column{0.4\textwidth}
    \begin{columns}
      \centering
      \column{0.5\textwidth}
      \centering
      \xincludegraphics[width=0.9\textwidth]{./TextureMapsImages/GLRLM_46_left_grayscale.png}\\
      \vspace{0.3mm}
      \xincludegraphics[width=0.9\textwidth, label=b, labelbox]{./TextureMapsImages/GLRLM_46_LEFT_1_crop.png}\\
      \vspace{0.3mm}
      \xincludegraphics[width=0.9\textwidth, label=d, labelbox]{./TextureMapsImages/GLRLM_46_LEFT_3_crop.png}
      \column{0.5\textwidth}
      \centering
      \xincludegraphics[width=0.9\textwidth, label=a, labelbox]{./TextureMapsImages/GLRLM_46_LEFT_0_crop.png}\\
      \vspace{0.3mm}
      \xincludegraphics[width=0.9\textwidth, label=c, labelbox]{./TextureMapsImages/GLRLM_46_LEFT_2_crop.png}\\
      \vspace{0.3mm}
      \xincludegraphics[width=0.9\textwidth, label=e, labelbox]{./TextureMapsImages/GLRLM_46_LEFT_4_crop.png}
    \end{columns}
  \end{columns}
\end{frame}

\begin{frame}{Bone Morphometry}

  \begin{columns}[onlytextwidth]
    \column{0.55\textwidth}
    Bone morphometry features describe the 3D shape of subchondral bone trabecular structure.\newline
    Five bone morphometry features:
    \begin{itemize} \itemsep0.5em
      \item Bone volume density (Bv/Tv) \textbf{[a]}
      \item Bone surface density (Bs/Bv) \textbf{[b]} 
      \item Trabecular thickness (Tb.Th) \textbf{[c]} 
      \item Trabecular separation (Tb.Sp) \textbf{[d]} 
      \item Trabecular number (Tb.N) \textbf{[e]} 
    \end{itemize}
    \column{0.4\textwidth}
    \begin{columns}
    \centering
    \column{0.5\textwidth}
    \centering
    \xincludegraphics[width=0.9\textwidth]{./TextureMapsImages/BM_46_LEFT_grayscale.png}\\
    \vspace{0.3mm}
    \xincludegraphics[width=0.9\textwidth, label=b, labelbox]{./TextureMapsImages/BM_46_LEFT_1_crop.png}\\
    \vspace{0.3mm}
    \xincludegraphics[width=0.9\textwidth, label=d, labelbox]{./TextureMapsImages/BM_46_LEFT_3_crop.png}
    \column{0.5\textwidth}
    \centering
    \xincludegraphics[width=0.9\textwidth, label=a, labelbox]{./TextureMapsImages/BM_46_LEFT_0_crop.png}\\
    \vspace{0.3mm}
    \xincludegraphics[width=0.9\textwidth, label=c, labelbox]{./TextureMapsImages/BM_46_LEFT_2_crop.png}\\
    \vspace{0.3mm}
    \xincludegraphics[width=0.9\textwidth, label=e, labelbox]{./TextureMapsImages/BM_46_LEFT_4_crop.png}
    \end{columns}
  \end{columns}
  \note{
    \begin{itemize}
      \item Bv/Tv stands for Bone Volume over Total Volume.
      \item Bs/Tv stands for Bone Surface over Bone Volume.
    \end{itemize}
    The computation of those features is based on:
    \begin{itemize}
      \item Ntotal is the total number of voxels in the volume of interest.
      \item Nbone represents the number of voxels in the bone area.
      \item Nboundary is the number of voxels that are part of the bone/non-bone boundary in the volume of interest
    \end{itemize}
    In order to obtain those numbers, a segmentation of the image highlighting the bone is needed (can be done thanks to a simple threshold)
  }
\end{frame}

{
\setbeamercolor{background canvas}{bg=kitwarewhite}
\begin{frame}{Texture Rendering}
  \begin{center}
  \includegraphics[width=0.6\textwidth]{./figures/FigureRunLengthNonUniformity_withVolume}\\
  \end{center}
  \vspace{0.1cm}
  \centering
  \note{
  Run-length non-uniformity texture map.
a) right (unaffected) femur, b) left (injured) femur.\\
First column: microCT slice intensity. Second column: corresponding texture map (note that texture is computed on the 3D image, and that texture values in purple clearly identify injury-induced mass.).\\
Third column: texture image rendering.}
\end{frame}
}

\begin{frame}{Shape Analysis}
  \begin{center}
  \includegraphics[width=0.9\textwidth]{./figures/analysis_shape.png}\\
  \end{center}
  \vspace{0.1cm}
  \centering
  Shape analysis results for all subjects, including superimposed 3D models (left) and visualization of pathological shape changes via signed distances (right).
  \note{
    Iterative closest point (ICP) to quantitatively analyze shape differences between left and right limbs.\\
    ICP is performed by computing distances or 3D vector from each vertex of the injured (left) model to the unaffected (right) model.
  }
\end{frame}

{
\setbeamercolor{background canvas}{bg=kitwarewhite}
\section{Results}
\begin{frame}{Results}
  \centering
  \includegraphics[width=0.7\textwidth]{./figures/features_pcaBoneVSpcaTexture1000Bins.png}\\
  \vspace{0.1cm}
  First principal component of the bone morphometry analysis versus first principal component of the texture analysis.
\end{frame}
}

\section{Conclusion}

\begin{frame}{Conclusions}
  \begin{itemize} \itemsep1.0em

    \item Computed biomarkers based on not only traditional bone morphometrics, but also texture and shape-based metrics.
  \end{itemize}

  \begin{columns}[onlytextwidth]
    \column{0.75\textwidth}
    \begin{itemize} \itemsep1.0em
      \item High performance implementation in ITK. N-dimensions. Local Texture Maps.
    \end{itemize}
    \column{0.25\textwidth}
    \centering
    \includegraphics[width=0.8\textwidth]{./logos/logo_ITK.png}
  \end{columns}

  \begin{itemize} \itemsep1.0em
    \item Workflow can be adapted to target other modalities and pathologies.
    \item We were able to classify injured and healthy femurs with simple statistical methods.
  \end{itemize}
\end{frame}

{
\setbeamercolor{background canvas}{bg=kitwarewhite}
\begin{frame}[fragile]{Performance Comparison: ITK vs Matlab (2D)}
  \centering
  \includegraphics[height=0.9\textheight]{./figures/TextureFeaturesPerformanceComparison.png}
\end{frame}

\begin{frame}{Open Source Development}
  \begin{columns}[onlytextwidth]
    \column{0.80\textwidth}
    \metroset{block=fill}
    \begin{block}{All analysis filters were implemented as part of \textbf{ITK}, the Insight Toolkit:}
    \begin{itemize} \itemsep1.0em
      \item High performance, greatly reducing computational time in texture analysis.
      \item Able to compute 3D maps of multidimensional features
    \end{itemize}
  \end{block}
    \column{0.20\textwidth}
    \centering
    \includegraphics[width=0.8\textwidth]{./logos/logo_ITK.png}
  \end{columns}
  \vspace{0.2cm}
  \begin{columns}[onlytextwidth]
    \column{0.80\textwidth}
    \metroset{block=fill}
    \begin{block}{\textbf{Python} packages available for Windows, Mac and Linux}
    % \noindent \textbf{Python} packages available for Windows, Mac and Linux
      \centering
      pip install itk-texturefeatures itk-bonemorphometry
  \end{block}
    \column{0.20\textwidth}
    \centering
    \includegraphics[width=0.99\textwidth]{./logos/logo_python.png}
  \end{columns}
  \vspace{0.2cm}
  \begin{columns}[onlytextwidth]
    \column{0.80\textwidth}
    \metroset{block=fill}
    \begin{block}{Graphical interface in 3DSlicer: Bone Texture Extension}
    \end{block}
    \column{0.20\textwidth}
    \centering
    \includegraphics[width=0.65\textwidth]{./logos/logo_slicer_horizontal.png}\hspace{0.1mm}
    \includegraphics[width=0.3\textwidth]{./logos/logo_BoneTexture.png}
  \end{columns}
  \note{
    All the results presented here were computed with newly developed tools:
    \begin{itemize}
      \item Able to compute 3D maps of multidimensional features at a reduced computational cost
      \item developed as part of ITK: open source and free to use
    \end{itemize}
    Available to use in python
    \begin{itemize}
      \item python package
      \item easy/quick to install and use
    \end{itemize}
    User friendly interface available in 3DSlicer (also free and open source):
    \begin{itemize}
      \item extension called Bone Texture Extension
    \end{itemize}
  }
\end{frame}

{
\setbeamercolor{background canvas}{bg=kitwarewhite}
\begin{frame}{Medical Computing}
    \centering
    \includegraphics[width=0.9\textwidth]{./logos/kitware_medical_HQ.png}
\end{frame}
}

\begin{frame}[standout]
  \centering
  {\huge Thanks!}\\
  pablo.hernandez@kitware.com\\
  kitware@kitware.com
  \vspace{1cm}
  % {\Huge Questions?}\\
\end{frame}

\appendix

{
\setbeamercolor{background canvas}{bg=kitwarewhite}
\begin{frame}[fragile]{GLCM Matrices}
  \centering
  \includegraphics[width=0.9\textwidth]{./figures/GLCM_matrices.png}
\end{frame}

\begin{frame}[fragile]{GLRLM Matrices}
  \centering
  \includegraphics[width=0.9\textwidth]{./figures/GLRLM_matrices.png}
\end{frame}
}


\begin{frame}[allowframebreaks]{References}

  %\bibliography{references}
  %\bibliographystyle{abbrv}
  \printbibliography

\end{frame}


\end{document}

